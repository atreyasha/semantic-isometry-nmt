% load relevant packages
\documentclass[11pt,a4paper]{article}
\usepackage[hyperref]{acl2020}
\usepackage{booktabs}
\usepackage{makecell}
\usepackage{float}
\usepackage{tabularx}
\usepackage{times}
\usepackage{multirow}
\usepackage{lipsum}
\usepackage{amsmath}
\usepackage{amssymb}
\usepackage{mathtools}
\usepackage{latexsym}
\usepackage{graphicx}
\graphicspath{{../../img/}}
\renewcommand{\UrlFont}{\ttfamily\small}
\usepackage{microtype}
\aclfinalcopy
%\setlength\titlebox{5cm}
\newcommand\BibTeX{B\textsc{ib}\TeX}

% administrative details
\title{Investigating the isometric properties of Neural Machine Translation models on binary semantic-equivalence spaces}
\author{Atreya Shankar \\
  Cognitive Systems, University of Potsdam \\
  Department of Computational Linguistics, University of Zürich \\
  \texttt{atreya.shankar@\{uni-potsdam.de,uzh.ch\}}}
\date{\today}

% start document
\begin{document}

% produce title
\maketitle

% abstract
\begin{abstract}
  \textit{Isometry} is defined mathematically as a distance-preserving transformation between two metric spaces. In this research, we hypothesize that well-performing Neural Machine Translation (NMT) models function approximately isometrically on semantic metric spaces. That is to say, if two sentences are semantically equivalent on the source side, they should remain semantically equivalent after translation on the target side. We begin by utilizing two NMT models of varying performance to translate semantically-equivalent paraphrases based off diverse WMT19 test data references. In order to quantify and simplify the notion of a semantic metric space, we treat it as a probabilistic binary semantic-equivalence space indicating either semantic equality or inequality; achieved by fine-tuning three transformer-based language models on Google's PAWS-X paraphrase detection task. By using the paraphrase detection outputs, we investigate the frequency and composition of semantically isometric behaviour in the NMT models' inputs and outputs.
\end{abstract}

% body
\section{Introduction}

\textit{Isometry} is defined mathematically as a distance-preserving transformation between two metric spaces \cite{coxeter1961introduction}. In this research, we view Neural Machine Translation (NMT) models from the perspective of semantic isometry and hypothesize that well-performing NMT models function approximately isometrically on semantic metric spaces. That is to say, if two sentences are semantically equivalent on the source side, they should remain semantically equivalent after translation on the target side given a well-performing NMT model. A simplified illustration of isometry in higher dimensional functional spaces can be seen in Figure \ref{isometry_visual}.

In order to conduct our investigation, we start by acquiring semantically equivalent paraphrases of WMT19 legacy and additional test references from \citet{freitag-bleu-paraphrase-references-2020} for \texttt{en$\rightarrow$de}. Next, we utilize two NMT models of varying performance, specifically the SOTA FAIR's WMT19 winning single transformer \cite{ng2019facebook} and the non-SOTA Scaling NMT WMT16 Transformer \cite{ott2018scaling}, in order to translate the aforementioned paraphrases in the \texttt{de$\rightarrow$en} translation direction. We use the former model pre-trained from \texttt{fairseq} \cite{ott2019fairseq} and train the latter model from scratch.

\begin{figure}
  \centering
  \includegraphics[trim={1.0cm 0cm 0cm 1.0cm},clip,width=0.52\textwidth]{isometry_visualized.png}
  \caption{Illustration of isometry in higher dimensional functional transformations \citep{Hegde-Numax}}
  \label{isometry_visual}
\end{figure}

Next, we utilize three well-performing paraphrase detection models to approximate isometry in the NMT models' translations. These paraphrase detection models are based off the mBERT\textsubscript{Base} \cite{devlin-etal-2019-bert}, XLM-R\textsubscript{Base} \cite{conneau2019unsupervised} and XLM-R\textsubscript{Large} \cite{conneau2019unsupervised} pre-trained multilingual language models; which are correspondingly fine-tuned on Google's PAWS-X paraphrase detection task \cite{pawsx2019emnlp, hu2020xtreme}.
 
Using the outputs of the paraphrase detection models, we finally investigate the frequency and composition of semantically isometric behaviour in the NMT models' inputs and outputs. We release our latest models and source code in our public GitHub repository\footnote{\url{https://github.com/atreyasha/semantic-isometry-nmt}}.

% 1. Introduction
% TODO add post-paper contributions as concretely as possible -> add into abstract as well
% TODO state expectations more concretely here and in abstract
% TODO state hypothesis more clearly that isometry is positively correlated with general performance of a model -> perhaps show this statistically where possible with appropriate statistical test
% TODO think of title and how this links up with everything -> change it to reflect content of paper if necessary
% TODO make abstract more similar to introduction where possible
% TODO add github repository and any other important links inside paper
% add information of chrF_2 comparisons/tests
% look up fairseq API and whether this is correct usage
% fix "based off" and change to something else to get diversity

\section{Isometry and approximations}

The concept of isometry in the context of semantic metric spaces can be \textit{exactly} expressed as follows; where $s_i \in \mathbb{R}^{V \times N}$ refers to an input sentence's tokenized matrix form for vocabulary size $V$ and maximum sentence length $N$, $f: \mathbb{R}^{V \times N} \to \mathbb{R}^{V' \times N'}$ refers to the NMT model's inference function and $D_L: \mathbb{R}^{V \times 2N} \to \mathbb{R}_+$ refers to a semantic distance metric function for language $L$ corresponding to the language of the respective sentences:

\begin{equation}  
  \label{exact_isometry_eqn}
  D_X(s_1,s_2) = D_Y(f(s_1),f(s_2))
\end{equation}

While elegant, this representation of isometry and a semantic distance metric is problematic for two key reasons.

\begin{enumerate}
\item Exact isometry may not be a practical condition to achieve given real-life data instances with stochastic noise.
\item Constructing continuous semantic metric spaces from discrete textual data is a difficult task and is in itself a developing field of research \cite{cer2017semeval, michel2019evaluation}.
\end{enumerate}

\subsection{Approximate isometry}

To address the first issue, we loosen the constraints of exact isometry to \textit{approximate} isometry:

\begin{equation} 
  \label{approx_isometry_eqn}
  D_X(s_1,s_2) \approx D_Y(f(s_1),f(s_2)) 
\end{equation}

With this approximation, we can simplify the isometric relationship further into a binary semantic-equivalence function $S_L: \mathbb{R}^{V \times 2N} \to \{0,1\}$, which compresses semantic distance metrics to semantic equality ($S_L=1$) or inequality ($S_L=0$) depending on some variable threshold $\delta_L \in \mathbb{R}_+$:

\begin{equation}
  \label{bounded_isometry_eqn}
  S_L(s_1,s_2) =
  \begin{cases}
    1, &D_L(s_1,s_2) \leq \delta_L \\
    0, &D_L(s_1,s_2) > \delta_L
  \end{cases}
\end{equation}

It is worth noting that the formulation in $S_L$ is more meaningful for inferring isometry from semantic equality than from semantic inequality, due to the presence of a tighter bound for the former than the latter.

\subsection{Probabilistic semantic-equivalence spaces}

To address the second issue, we effectively delegate away the actual computation of a semantic distance metric and convert this into a probabilistic process; with a new definition for $S_L$ below given a probability threshold $\epsilon$ with a typical value of 0.5. This reformulation allows for the utility of statistical paraphrase detection models without explicit computation of semantic metric spaces. 

\begin{equation}
  \label{bounded_isometry_probability_eqn}
  S_L(s_1,s_2) =
  \begin{cases}
    1, &P\big(D_L(s_1,s_2) \leq \delta_L\big) \geq \epsilon \\
    0, &P\big(D_L(s_1,s_2) \leq \delta_L\big) < \epsilon
  \end{cases}
\end{equation}

With the aforementioned simplifications, we now re-write our equation for approximate isometry as follows:

\begin{equation}  
  \label{exact_approx_isometry_eqn}
  S_X(s_1,s_2) = S_Y(f(s_1),f(s_2))
\end{equation}

% 2. Motivation (or more thematic title)
% TODO extrapolate respective changes from here to all sections with terminology
% TODO consider changing paraphrase score to distance metric -> depends on how the research is presented
% TODO consider changing name to semantic-similarity or paraphrase-detection space -> think of how to not make this misleading and how to not confuse discete and continuous semantic spaces
% TODO consider calling analysis semantic-equivalence-step functions or probabilistic semantic-equivalence, or binary might be good enough -> consider getting rid of metric spaces if they are not useful
% think more about whether to include or exclude adversarial term since this might be a grey area -> qualify various means of being adversarial ie. targetted through model or perhaps just an intention

\section{Related work}

Based on a survey of recent literature in Natural Language Processing (NLP) and NMT, we were unable to find explicitly similar studies to our research. However, we would argue that the closest field in NLP to this research would be \textit{adversarial paraphrasing}.

\citet{michel2019evaluation} describes adversarial paraphrasing in the purview of machine translation as constructing paraphrases that are \textit{``meaning preserving on the source-side, but meaning-destroying on the target-side''}. For the sake of comparison, we would mildly \textit{paraphrase} this description of adversarial paraphrasing to \textit{``the process of perturbing an input sentence such that it is semantically equivalent on the source-side, but semantically inequivalent on the target-side"}.

In this sense, the study of adversarial paraphrasing in machine translation could be interpreted as a targetted probe into semantic \textit{anisometry} of NMT models, compared to our research which would be an untargetted probe into semantic isometry of NMT models. Therefore adversarial paraphrasing, while having the opposite intent, is still highly similar to our research.

\paragraph{\citet{michel2019evaluation}:} This research lays out the framework for evaluating adversarial perturbations in sequence-to-sequence models. Additonally, this research compared three automatic sequence evaluation metrics, specifically BLEU \cite{papineni2002bleu}, METEOR \cite{denkowski2014meteor} and chrF$_2$ \cite{popovic2015chrf}, against human judgment for evaluating semantic similarity. Results from their experiments showed that chrF$_2$ correlates best out of the three similarity metrics with human judgment for semantic similarity detection. We utilize this finding in later parts of our study and attempt to compare the outputs of our paraphrase detection models with respective chrF$_2$ scores.

\paragraph{\citet{fadaee2020unreasonable}:} This research lays out a simple framework for constructing adversarial paraphrases through logical operations such as word insertion/deletion and numerical/gender substitution. The research correspondingly showed that such minor modifications could lead to disproportionately larger changes in translation outputs; thereby showing an adversarial effect. This research ultimately claimed that modern NMT models are generally \textit{volatile}, or vulnerable, to targetted adversarial attacks. We attempt to compare this claim with our findings in later parts of this study. 

% 3. Related work
% TODO consider removing comparisons to own study at the end of related work descriptions
% much work has been done on paraphrase detection but not neccessarily linking this directly to translation models -> would be an interesting link to NMT model evaluation

\section{Experimental setup}

\subsection{Data sets}

\subsubsection{WMT19 en-de references and corresponding paraphrases}

\citet{freitag-bleu-paraphrase-references-2020} builds on the premise that while automatic evaluation metrics, such as BLEU, are important for NMT model evaluation; the presence of diverse translation references is also critical. Motivated by the observation that typical references show poor diversity, \citet{freitag-bleu-paraphrase-references-2020} focuses on two goals; namely creating additional high quality WMT19 test references, as well as paraphrasing both existing (or legacy) and additional WMT19 test references in the \texttt{en$\rightarrow$de} translation direction. These services were ultimately rendered by a professional translation service using different sets of linguists for different tasks to reduce systematic bias.

While these additional references serve the purpose of diversifying evaluation references, we see them as a source of high-quality semantically equivalent \texttt{de} paraphrases with varied lexical and syntactical features. Below are the key \texttt{de} data sets for that were used in our research, which were originally designed to be references for the \texttt{en$\rightarrow$de} translation direction.

\paragraph{WMT19 legacy test references:} This refers to the existing \texttt{newstest2019} translation references with 1997 sentences. For abbrevation purposes, we refer to this data set as \texttt{WMT}. 

\paragraph{WMT19 additional test references:} This refers to additional references produced as a result of \citet{freitag-bleu-paraphrase-references-2020} with 1997 sentences. For abbrevation purposes, we refer to this data set as \texttt{AR}. 

\paragraph{WMT19 legacy test paraphrased references:} This refers to the paraphrased version of the existing \texttt{newstest2019} translation references produced as a result of \citet{freitag-bleu-paraphrase-references-2020} with 1997 sentences. For abbrevation purposes, we refer to this data set as \texttt{WMT.p}. 

\paragraph{WMT19 additional test paraphrased references:}This refers to the paraphrased version of the additional translation references produced as a result of \citet{freitag-bleu-paraphrase-references-2020} with 1997 sentences. For abbrevation purposes, we refer to this data set as \texttt{AR.p}.

For brevity, we concatenate the aforementioned data sets into WMT19 Legacy and WMT19 AR. Both data sets have 1997 pairs of semantically equivalent \texttt{de} sentences or paraphrases.

\vspace{-10pt}
\begin{align}
  \text{WMT19 Legacy} &= \{\text{WMT} \cup \text{WMT.p} \} \label{wmt19legacy} \\
  \text{WMT19 AR} &= \{\text{AR} \cup \text{AR.p} \} \label{wmt19ar}
\end{align}

\subsubsection{PAWS-X}

PAWS-X is a cross-lingual adversarial data set for paraphrase identification released by Google Research \citep{pawsx2019emnlp}. PAWS-X stems originally from the PAWS data set released by \citet{zhang2019paws} which is an abbreviation for \textbf{P}araphrase \textbf{A}dversaries from \textbf{W}ord \textbf{S}crambling.

The original motivation behind the PAWS data set was that existing paraphrase detection data sets lacked non-paraphrase sentence pairs with high lexical overlap. The PAWS data set was therefore released to drive progress in creating models that utilize fine-grained structure and context of sentence pairs.

\begin{figure*}
  \centering 
  \includegraphics[trim={0cm 0cm 0cm 0cm},clip,width=\textwidth]{transformer_nmt_evolution.pdf}
  \caption{Training and validation cross entropy loss for Scaling NMT WMT16 Transformer}
  \label{transformer_nmt_evolution}
\end{figure*}

The PAWS data set contains 108,463 paraphrase and non-paraphrase sentence pairs with high lexical overlap. These sentence pairs were bulk sourced from Wikipedia and Quora Question Pairs; followed by controlled word swapping and back translation to create challenging sentence pairs for paraphrase detection. The generated sentence pairs were finally evaluated for fluency and general quality by human raters.

As noted in \citet{pawsx2019emnlp}, one limitation of adversarially generated data sets such as PAWS is their pre-dominant focus on the English language. In order to address this issue, \citet{pawsx2019emnlp} released PAWS-X; which consists of 23,659 human translated evaluation sentence pairs and 296,406 machine-translated training sentence pairs derived from the Wikipedia subset of the original PAWS data set. These sentence pairs were translated from English to six typologically distinct languages; namely French, Spanish, German, Chinese, Japanese and Korean.

The release of PAWS-X provides many advantages to the field of NLP, particularly the creation of a new benchmark to promote research in multilingual and zero-shot paraphrase detection. This can already be seen by the incorporation of PAWS-X into Google's recent Cross-lingual TRansfer Evaluation of Multilingual Encoders (XTREME) benchmark system \cite{hu2020xtreme}.

\subsubsection{WMT16 de-en}

In this study, we replicate a non-SOTA NMT model from scratch based off the Scaling NMT WMT16 workflow \cite{ott2018scaling}. While the original implementation in \citet{ott2018scaling} is based on the \texttt{en$\rightarrow$de} translation direction, our implementation trains a NMT model in the reverse translation direction; specifically \texttt{de$\rightarrow$en}.

For this, we use WMT16 \texttt{de$\rightarrow$en} training data with 4.5M sentence pairs, \texttt{newstest2013} as our validation set and \texttt{newstest2014} as our test set. We utilize a vocabulary of 32K symbols based off a joint source and target byte-pair encoding (BPE; \citealt{sennrich2015neural}).   

% 4.1. Data sets
% TODO check consistent spelling of data sets across paper
% TODO be clear about translation direction -> check for consistency throughout paper on direction
% explain why AR was used, to provide additional paraphrases on top of legacy data since legacy WMT19 test data was shown to be poorly sourced -> talk about paraphrasing done very strongly which is why some were zeros on the source side already
% bring up actual examples of paraphrases and detection tasks -> to give some perspective

\subsection{Models}

\subsubsection{FAIR WMT19 Transformer}

We utilize FAIR's winning WMT19 single Transformer model as our SOTA NMT model. Focusing particularly on the \texttt{de$\rightarrow$en} translation direction, the FAIR WMT19 Transformer was the top performing model in WMT19 with a SacreBLEU \cite{post-2018-call} score of 40.8.

As per \citet{ng2019facebook}, the key factors that led to SOTA performance include \texttt{langid} filtering of crawled bitext data, large-scale back translation as a form of data augmentation and noisy channel model reranking. We utilized this model directly from the \texttt{fairseq} API \cite{ott2019fairseq}. 

\subsubsection{Scaling NMT WMT16 Transformer}

We replicate the Scaling NMT WMT16 Transformer based on \citet{ott2018scaling} by training it from scratch. However, we swap the translation direction from \texttt{en$\rightarrow$de} to \texttt{de$\rightarrow$en}; such that we can ultimately use this model to translate WMT19 paraphrases from \texttt{de$\rightarrow$en}. We intentionally choose this workflow since it would produce a non-SOTA transformer which would be useful for us downstream to introduce performance-dependent variance in the translation of WMT19 paraphrases.

\begin{figure*}
  \centering 
  \includegraphics[trim={0.7cm 0cm 0cm 0cm},clip,width=\textwidth]{paraphrase_detection_models_evolution.pdf}
  \caption{Training loss and validation accuracy w.r.t. training steps for paraphrase detection models}
  \label{paraphrase_detection_model_evolution}
\end{figure*}

Besides the aforementioned modification, we follow the same setup as per \citet{ott2018scaling}. Specifically, we use a ``big'' transformer model based off \citet{vaswani2017attention}; with 6 blocks in the encoder and decoder networks. This model has a total of 210M parameters.

During training, we apply dropout \cite{srivastava2014dropout} with probability 0.3 and utilize the Adam optimizer \cite{kingma2014adam} with $\beta_1 = 0.9$ and $\beta_2=0.98$. We use a learning rate schedule where the learning rate increases linearly for 4,000 steps from 1e-7 until 1e-3. The learning rate then decays proportionally to the inverse square root of the number of training steps. We utilize label smoothing with weight 0.1 for the uniform prior distribution over the vocabulary \cite{pereyra2017regularizing}. We use large batch sizes with the maximum number of tokens per batch being 7168. Furthermore, we apply gradient accumulation for 8 steps before updating the model; which is known as \texttt{update-freq} in the \texttt{fairseq} API. We also exploit \texttt{fairseq}'s half precision floating point (FP16) functionality for more efficient training.

Finally, we train this model for 6 days on a single NVIDIA Tesla-V100 16GB GPU. During training, we monitor the validation loss and enable checkpoint-saving for the best performing checkpoint on the validation set. We train the model up until $\sim$285K updates.

Our best performing checkpoint was saved at $\sim$180K updates as seen in Figure \ref{transformer_nmt_evolution}. For evaluation on the test set, we utilize beam search with a beam width of 5. Our final Scaling NMT WMT16 Transformer achieved a SacreBLEU \cite{post-2018-call} score\footnote{\footnotesize SacreBLEU signature:\\BLEU+case.mixed+lang.de\nobreakdash-en+numrefs.1+smooth.exp+\\test.wmt14/full+tok.13a+version.1.4.12} of 31.0 on the \texttt{newstest2014} test set.

\subsubsection{Paraphrase detection models}

As noted in equation \ref{bounded_isometry_probability_eqn}, paraphrase detection models are useful in computing probabilistic semantic-equivalence spaces, or otherwise the $S_L$ function. We follow a similar framework as that detailed in Google's XTREME benchmark \cite{hu2020xtreme} and fine-tune pre-trained multilingual transformer language models on the PAWS-X paraphrase detection task. We focus specifically on three multilingual transformer language models, specifically mBERT\textsubscript{Base} (104 languages; 172M parameters; \citealt{devlin-etal-2019-bert}), XLM-R\textsubscript{Base} (100 languages; 270M parameters; \citealt{conneau2019unsupervised}) and XLM-R\textsubscript{Large} (100 languages; 550M parameters; \citealt{conneau2019unsupervised}) using HuggingFace's \texttt{transformers} library \cite{Wolf2019HuggingFacesTS} with model variants optimized for sequence classification.

While our implementation is similar to that of Google's XTREME benchmark, we modify some aspects of the workflow to suit our needs. Most importantly, we fine-tune our multilingual language models on PAWS-X training data from all 7 languages instead of only English in order to reap the benefits of diverse multilingual data.

\begin{table}
  \centering
  \begin{tabular}{llll}
    \hline
    \textbf{Language} & \textbf{mBERT\textsubscript{B}} & \textbf{XLM-R\textsubscript{B}} & \textbf{XLM-R\textsubscript{L}} \\
    \hline
    en & 0.940 & 0.946 & 0.960 \\
    de & 0.898 & 0.900 & 0.912 \\
    es & 0.908 & 0.922 & 0.928 \\
    fr & 0.922 & 0.917 & 0.933 \\
    ja & 0.836 & 0.836 & 0.859 \\
    ko & 0.841 & 0.847 & 0.870 \\
    zh & 0.854 & 0.861 & 0.876 \\
    \hline \hline
    $\mu$ & 0.886 & 0.890 & \textbf{0.906} \\
    \hline
  \end{tabular} 
  \caption{Language-specific summary of macro-F\textsubscript{1} scores of paraphrase detection models on the PAWS-X test set; languages are abbreviated based on ISO 639-1; B and L refer to base and large respectively}
  \label{pawsx_score_breakdown}
\end{table}

For all models, we enforce a maximum sequence length of 128 tokens since PAWS-X sentence pairs generally fit it into this range. We use a modified Adam optimizer with decoupled weight decay regularization \cite{DBLP:journals/corr/abs-1711-05101} with $\beta_1 = 0.9$ and $\beta_2=0.999$. We train all models for 10 epochs or $\sim$110K updates with a global batch size of 32. We also use a linearly decaying learning rate schedule without warmup steps. Lastly, we monitor accuracy on the PAWS-X validation set for all languages in order to determine the best performing checkpoint.

Specific to mBERT\textsubscript{Base} and XLM-R\textsubscript{Base}, we use a batch size of 32 without gradient accumulation and an initial learning rate of 2e-5. As for XLM-R\textsubscript{Large}, we use an initial learning rate of 1e-6 and local batch size of 8 with 4 gradient accumulation steps to curb GPU out-of-memory (OOM) issues.

We fine-tune mBERT\textsubscript{Base}, XLM-R\textsubscript{Base} and XLM-R\textsubscript{Large} for 14 hours, 15 hours and 2.5 days on a single NVIDIA Geforce GTX 1080-Ti 12GB GPU respectively. The best checkpoints are achieved and saved at $\sim$20K, $\sim$80K and $\sim$40K updates respectively, as seen in Figure \ref{paraphrase_detection_model_evolution}.

As seen in Table \ref{pawsx_score_breakdown}, all three models perform well especially on our target languages of \texttt{en} and \texttt{de}. Overall, the best performing model on the PAWS-X test set is XLM-R\textsubscript{Large} with a macro-F\textsubscript{1} of 0.906.

% 4.2. Models
% check and refine GPU naming scheme, ie. with copyright and where to put dashes

\subsection{Evaluation protocols}

Given our WMT19 Legacy and WMT19 AR \texttt{de} paraphrase datasets defined in equations \ref{wmt19legacy} and \ref{wmt19ar}, we translate all pairs of paraphrases using both the FAIR WMT19 and the Scaling NMT WMT16 Transformers in the \texttt{de$\rightarrow$en} translation direction. With this, we have the source \texttt{de} paraphrases along with their target-side \texttt{en} translations. We use these samples for further investigation. 

\subsubsection{Isometry on binary semantic-equivalence spaces}

\paragraph{Vector representation:} We modify our representation of the $S_L$ relations in equation \ref{exact_approx_isometry_eqn} in order to have a more concise vectorized form of the relationship. We assign the probability threshold $\epsilon$ from equation \ref{bounded_isometry_probability_eqn} a constant value of 0.5 for all computations of $S_L$ in this research.

\vspace{-5pt}
\begin{gather}
  \mathbf{S_{XY}} = \begin{bmatrix} S_X(s_1, s_2) \\[5pt] S_Y(f(s_1), f(s_2)) \end{bmatrix} \\[10pt] 
  \mathbf{S_{XY}^{\mathsf{T}}} = \begin{bmatrix} S_X(s_1, s_2) & S_Y(f(s_1), f(s_2)) \end{bmatrix}
\end{gather}

\paragraph{Multi-model decisions:} Given this formulation of $\mathbf{S_{XY}^{\mathsf{T}}}$, it is worth noting that each of the three paraphrase detection models would compute a separate $\mathbf{S_{XY}^{\mathsf{T}}}$ term. Since our paraphrase detection models were evaluated to have a non-zero (albeit small and similar) error rates, we decide to use the $\mathbf{S_{XY}^{\mathsf{T}}}$ outputs of all three models and compute the statistical mode of the three outputs. Such a majority decision would provide more confidence in any particular decision from the models. For simplicity, we assign the statistical mode function with the symbol $\mathbf{M}$. We therefore compute $\mathbf{M(S_{XY}^{\mathsf{T}}})$ to check for a majority decision over the three paraphrase detection models. We assign the empty set $\emptyset$ in case no majority decision exists.

\paragraph{Discrete possibilities:} Since the output of $S_L$ falls in the binary set of $\{0,1\}$, we can effectively compute all five possibilities of the consequent majority decision $\mathbf{M(S_{XY}^{\mathsf{T}}})$. We describe these below.

\begin{figure*}
  \centering 
  \includegraphics[trim={0cm 0cm 0cm 0cm},clip,width=\textwidth]{paraphrase_detection_softmax_all.pdf}
  \caption{Normalized contour densities for target paraphrase softmax score against source paraphrase softmax score; grouped by NMT models (top), paraphrase detection models (top) and input data sets (right)}
  \label{paraphrase_detection_softmax_all}
\end{figure*}

\begin{enumerate}
\item $\mathbf{M(S_{XY}^{\mathsf{T}}}) = [1, 1]$: Both the source and target pairs of sentences were evaluated to be paraphrases and are therefore semantically equivalent on both sides. We consider this as approximately \textit{isometric behaviour}.
\item $\mathbf{M(S_{XY}^{\mathsf{T}}}) = [0, 0]$: Both the source and target pairs of sentences were evaluated to not be paraphrases and are therefore not semantically equivalent on both sides. While this does qualify as approximate isometry according to equation \ref{exact_approx_isometry_eqn}, we also observe this scenario is less conclusive for determining isometry compared to $\mathbf{M(S_{XY}^{\mathsf{T}}}) = [1, 1]$ because of a looser bound associated with $S_L=0$ as per equation \ref{bounded_isometry_eqn}. We therefore assume this behaviour to be \textit{ambiguous}.  
\item $\mathbf{M(S_{XY}^{\mathsf{T}}}) = [0, 1]$: The source sentences are evaluated to not be paraphrases while the target sentences are evaluated to be paraphrases. This implies that translation resulted in the sentence pair becoming semantically equivalent while they were not before. This qualifies as approximately anisometric behaviour and could be an interesting scenario to investigate further. We define this as \textit{type-1 anisometric behaviour}.
\item $\mathbf{M(S_{XY}^{\mathsf{T}}}) = [1, 0]$: The source sentences are evaluated to be paraphrases while the target sentences are evaluated to not be paraphrases. This implies that translation resulted in the sentence pair becoming semantically inequivalent while they were not before. This qualifies as approximately anisometric behaviour and could imply weak performance of the model on one of the sentences; or possibly some adversarial paraphrasing. We define this as \textit{type-2 anisometric behaviour}.
\item $\mathbf{M(S_{XY}^{\mathsf{T}}}) = \emptyset$: This implies that all models had different decisions and therefore no majority decision was reached. We assume this behaviour to be \textit{ambiguous}. 
\end{enumerate}

\paragraph{Frequency analysis:} Given the five aforementioned possibilties of $\mathbf{M(S_{XY}^{\mathsf{T}}})$, we measure the frequency of each possibility in each model's input and output sentences. This give us an insight into the isometric behaviour of each model and would also allow us to compare both models.

\paragraph{Sentence-level analysis:} Type-1 and type-2 anisometric behaviours could prove interesting for further investigation. For such cases, we probe further and analyze sentences individually.

\subsubsection{Relationship between chrF$_2$ and semantic-equivalence}

We return to one of the observations from \citet{michel2019evaluation}, specifically that the chrF$_2$ automatic evaluation metric \cite{popovic2015chrf} tends to correlate most positively with human judgment of semantic similarity compared to BLEU \cite{papineni2002bleu} and METEOR \cite{denkowski2014meteor}. We attempt to use our investigation of isometry on semantic-equivalence spaces to further investigate the relationship between semantic equality and the chrF$_2$ score. We replicate the chrF$_2$ setup from \citet{michel2019evaluation} by using the default \texttt{sacrebleu} implementation \cite{post-2018-call} with a $n$-gram upper limit of 6 and $\beta$ value of 2

\paragraph{Non-commutativity of chrF$_2$:} While replicating the chrF$_2$ setup of \citet{michel2019evaluation}, we observe that this particular formulation with $\beta = 2$ results in the chrF$_2$ metric being non-commutative, where $s_1$ and $s_2$ are input sentences:

\begin{equation}
  \text{chrF}_{2}(s_1,s_2) \neq \text{chrF}_{2}(s_2,s_1)
\end{equation}

\begin{figure*}
  \centering 
  \includegraphics[trim={0cm 0cm 0cm 0cm},clip,width=\textwidth]{paraphrase_detection_joint_decision.pdf}
  \caption{Frequency distribution of $\mathbf{M(S_{XY}^{\mathsf{T}}})$ by NMT models (top) and input data sets (right); filling colors indicate finer details on the type of majority decision}
  \label{paraphrase_detection_joint_decision}
\end{figure*}

Non-commutativity is an emergent property of chrF$_2$ with $\beta = 2$ since this would assign two times more weight to recall than precision \cite{popovic2015chrf}; which places an internal bias on input order. While this non-commutative formulation of chrF$_2$ would be useful for evaluating NMT models with explicit hypotheses and references, this would not be optimal as a semantic similarity metric since one would expect a semantic similarity metric to be commutative and unbiased towards input order.

\paragraph{Commutative variant of chrF$_2$:} As an alternative, we simulate commutativity by averaging the chrF$_2$ values for both input orders and introduce $\overline{\text{chrF}_2}$ as a commutatitve variant for chrF$_2$:

\begin{gather}
  \overline{\text{chrF}_2}(s_1,s_2) = \dfrac{\text{chrF}_2(s_1,s_2) + \text{chrF}_2(s_2,s_1)}{2} \\[5pt]
  \therefore \quad \overline{\text{chrF}_2}(s_1,s_2) = \overline{\text{chrF}_2}(s_2,s_1)
\end{gather}

We see this as a more optimal alternative than changing $\beta$ to 1 since this might veer further away from the experimental setup of \citet{michel2019evaluation}. Such commutative variants of automatic similarity metrics have also been considered for BLEU and METEOR in \citet{wieting-etal-2019-beyond} and were termed \textit{symmetric} instead of commutative.

\paragraph{Mapping $\overline{\text{chrF}_2}$ to $S_L$:}
We refer back to the majority decisions of the paraphrase detection models and remove all sentence pairs where $\mathbf{M(S_{XY}^{\mathsf{T}}}) = \emptyset$. We assume all remaining sentence pairs have been \textit{confidently} tagged by the paraphrase detection models. We assign the remaining sentence pairs with their respective $\overline{\text{chrF}_2}$ and $S_L$ values for the source and target sides. 

\paragraph{Correlation between $\overline{\text{chrF}_2}$ and $S_L$:} Finally, we replicate a similar statistical procedure as per \citet{michel2019evaluation} and compute the Pearson correlation coefficient $r_{xy}$ and the corresponding statistical significance $t$-test. For the $t$-test, we set the null hypothesis $H_0$ to be that there exists a non-positive correlation between $\overline{\text{chrF}_2}$ and $S_L$; while the alternative hypothesis $H_1$ is that there exists a positive correlation between $\overline{\text{chrF}_2}$ and $S_L$. We interpret the strength of correlations using guidelines from \citet{schober2018correlation}.     

% 4.3. Evaluation protocols
% 4.3.2. Correlation between chrF and semantic equality
% TODO state expectations about outcomes, some basic ideas -> can be done post-paper
% TODO consider refining and synchronizing model names as was done with data -> update plots
% TODO change phrasing of sentence-level analysis to show interest in all possibilities instead of just some

\section{Results}

\subsection{Isometry on binary semantic-equivalence spaces}

\subsubsection{Paraphrase detection softmax scores}

Figure \ref{paraphrase_detection_softmax_all} shows a normalized contour density estimate for paraphrase detection softmax scores. These scores are grouped by NMT models, input data sets and paraphrase detection models. We can observe that mBERT\textsubscript{Base} generally shows more variance in softmax scores compared to the XLM-R models. We can also observe that all models show more variance for translation outputs from the Scaling NMT WMT16 Transformer compared to those from the FAIR WMT19 Transformer. The softmax distributions are generally similar between the WMT19 Legacy and WMT19 AR input data sets.

\subsubsection{Frequency analysis}

Figure \ref{paraphrase_detection_joint_decision} shows a visual breakdown of absolute frequencies of $\mathbf{M(S_{XY}^{\mathsf{T}}})$ by NMT models and input data sets. Table \ref{isometry_frequency} shows a tabular breakdown of relative frequencies of $\mathbf{M(S_{XY}^{\mathsf{T}}})$ by NMT models and input data sets. Below are the key observations.

\begin{table*}[t!]
  \centering
  \begin{tabular*}{\textwidth}{c @{\extracolsep{\fill}} cccccc}
    \hline \\[-10pt]
    \multirow{2}[3]{*}{$\mathbf{M(S_{XY}^{\mathsf{T}}})$} & \multicolumn{3}{c}{\textbf{FAIR WMT19 Transformer}} & \multicolumn{3}{c}{\textbf{Scaling NMT WMT16 Transformer}} \\
    \cmidrule(lr){2-4} \cmidrule(lr){5-7}
    & WMT19 Legacy & WMT19 AR & $\mu$ & WMT19 Legacy & WMT19 AR & $\mu$ \\[3pt]
    \hline \hline \\[-10pt]
    $[1,1]$ & \textbf{0.698} & \textbf{0.686} & \textbf{0.692} & 0.554 & 0.541 & 0.548 \\
    $[0,0]$ & 0.091 & 0.104 & 0.097 & 0.153 & 0.154 & 0.153 \\
    $[0,1]$ & \textbf{0.074} & \textbf{0.065} & \textbf{0.069} & 0.033 & 0.031 & 0.032 \\
    $[1,0]$ & 0.018 & 0.030 & 0.024 & \textbf{0.133} & \textbf{0.130} & \textbf{0.131}\\
    $\emptyset$ & 0.120 & 0.117 & 0.118 & 0.128 & 0.144 & 0.136 \\
    \hline
  \end{tabular*}
  \caption{Relative frequency distribution for $\mathbf{M(S_{XY}^{\mathsf{T}}})$ by NMT models (top) and input data sets (top); $\mu$ indicates the macro-average of the relative frequency over the input data sets given a single model}
  \label{isometry_frequency}
\end{table*}

\begin{figure*}
  \centering 
  \includegraphics[trim={0cm 0cm 0cm 0cm},clip,width=\textwidth]{chrf_nmt.pdf}
  \caption{Distributions of source-side $\overline{\text{chrF}_2}$ for paraphrases in \texttt{de} and target-side $\overline{\text{chrF}_2}$ for paraphrases in \texttt{en} by NMT models (top) and input data sets (top)}
  \label{chrf_distribution}
\end{figure*}

\begin{enumerate}
\item \textbf{Isometry:} We observe a higher proportion of isometric behaviour with the FAIR WMT19 Transformer (69.2$\%$) compared to the Scaling NMT WMT16 Transformer (54.8$\%$).
\item \textbf{Type-1 anisometry:} We observe a higher proportion of type-1 anisometric behaviour with the FAIR WMT19 Transformer (6.9$\%$) compared to the Scaling NMT WMT16 Transformer (3.2$\%$).
\item \textbf{Type-2 anisometry:} We observe a lower proportion of type-2 anisometric behaviour with the FAIR WMT19 Transformer (2.4$\%$) compared to the Scaling NMT WMT16 Transformer (13.1$\%$).  
\item \textbf{Ambiguity:} We observe a lower proportion of ambiguous samples with the FAIR WMT19 Transformer (21.5$\%$) compared to the Scaling NMT WMT16 Transformer (28.9$\%$).
\item \textbf{Model agreement:} Based on Figure \ref{paraphrase_detection_joint_decision}, we observe that the majority of agreements are full agreements, followed by XLM-R\textsubscript{Base} and XLM-R\textsubscript{Large} agreements, mBERT\textsubscript{Base} and XLM-R\textsubscript{Base} agreements and finally mBERT\textsubscript{Base} and XLM-R\textsubscript{Large} agreements.
\end{enumerate}

\subsection{Correlation between $\overline{\text{chrF}_2}$ and $S_L$}

\subsubsection{Source and target $\overline{\text{chrF}_2}$ distributions}

Figure \ref{chrf_distribution} shows the distribution of $\overline{\text{chrF}_2}$ over the source (\texttt{de}) and target (\texttt{en}) sides, grouped over NMT models and input data sets. We can observe a larger variance of $\overline{\text{chrF}_2}$ points for the FAIR WMT19 Transformer outputs compared to those from the Scaling NMT WMT16 Transformer. Furthermore, we can observe a larger mean $\overline{\text{chrF}_2}$ value for the FAIR WMT19 Transformer compared to the Scaling NMT WMT16 Transformer. This can be inferred from the general distribution of points from the former being above the diagonal compared to those from the latter being centered near the diagonal.

\begin{figure*}
  \centering 
  \includegraphics[trim={0.25cm 0cm 0cm 0cm},clip,width=\textwidth]{chrf_paraphrase_detection_boxplot_joint_decision.pdf}
  \caption{Distribution of $\overline{\text{chrF}_2}$ against $S_L$ grouped by NMT models (top) and input data sets (top); *** indicates a statistically significant positive correlation between $\overline{\text{chrF}_2}$ and $S_L$ with $p \leq 0.001$ for the one-tailed $t$-test}
  \label{chrf_paraphrase_detection_joint_boxplot}
\end{figure*}

\begin{table*}[t!]
  \centering
  \begin{tabular*}{\textwidth}{c @{\extracolsep{\fill}} cccccc}
    \hline \\[-10pt]
    \multirow{2}[3]{*}{Statistic} & \multicolumn{3}{c}{\textbf{FAIR WMT19 Transformer}} & \multicolumn{3}{c}{\textbf{Scaling NMT WMT16 Transformer}} \\
    \cmidrule(lr){2-4} \cmidrule(lr){5-7}
    & WMT19 Legacy & WMT19 AR & $\mu$ & WMT19 Legacy & WMT19 AR & $\mu$ \\[3pt]
    \hline \hline \\[-10pt]
    $r_{xy}$ & 0.269 & 0.243 & \textbf{0.256} & 0.269 & 0.231 & \textbf{0.250} \\[3pt]
    $H_1: r>0$ & *** & *** & \textemdash & *** & *** & \textemdash \\[5pt]
    \hline \\[-10pt]
    Correlation & Weak & Weak & \textemdash & Weak & Weak & \textemdash \\[2pt]
    \hline
  \end{tabular*}
  \caption{Tabular summary of Pearson correlation coefficients $r_{xy}$, $t$-test alternative hypothesis and correlation strength interpretation \cite{schober2018correlation} grouped by NMT models (top) and input data sets (top); *** indicates $p\leq0.001$ for the one-tailed $t$-test}
  \label{chrf_correlation}
\end{table*}

\subsubsection{Correlation analysis}

Figure \ref{chrf_paraphrase_detection_joint_boxplot} shows the distribution of $\overline{\text{chrF}_2}$ against $S_L$ by over NMT models, input data sets and source-target origins. Table \ref{chrf_correlation} shows a breakdown of Pearson correlation coefficients $r_{xy}$ for $\overline{\text{chrF}_2}$ and $S_L$ by NMT models and input data sets.

Overall, we observe a significant positive correlation between $\overline{\text{chrF}_2}$ and $S_L$ with mean Pearson correlation coefficient $r_{xy}$ values of $0.256$ and $0.250$ for the FAIR WMT19 Transformer and the Scaling NMT WMT16 Transformer respectively. The one-tailed $t$-test used to ascertain significance showed a strongly significant positive correlation with $p\leq0.001$. According to the interpretation guidelines from \citet{schober2018correlation}, this range of $r_{xy}$ would imply a weak correlation strength between $\overline{\text{chrF}_2}$ and $S_L$. 

% 5. Results (facts)
% 5.1. Isometry analysis
% test <- aggregate(compressed_collection$Label,
% by = compressed_collection[c("model_name", "data_name","Label")],
% FUN=length)
% test_1 <- test
% test_1$x <- round(test_1$x/1997, 3)
% test_2 <- test
% test_2$x <- test_2$x/1997
% test_2 <- aggregate(test_2$x, by = test_2[c("model_name","Label")], FUN=mean)
% test_2$x <- round(test_2$x, 3)
% aggregate(compressed_collection$Label, by=compressed_collection["Type"], FUN=length)
% 5.2. chrF correlations
% groups <- split(hold[c("label", "value")], f = list(hold$model_name, hold$data_name))
% cor.test(as.numeric(as.character(groups[[1]]$label)), groups[[1]]$value, method="pearson", alternative="greater")

\section{Discussion}

% 6. Discussion (interpretation & comparisons)
% chrF correlation is much weaker in these cases compared to Michel et al. 2019 -> rough 1/2
% go into some examples for interpretations -> use examples of each category which can be interesting -> use actual examples instead of going into statistical tests
% interpretation of [0,0] and possible bad paraphrases from source-side -> criticism of WMT19 paraphrases
% look into statistical test between isometry and model performance in general
% hard to prove this unless we re-train the SOTA model without backtranslation -> possible future task
% contributions to Michel et al. 2019 analysis where possible
% go into discussion regarding chrF and semantic equality -> try to make statistical claims where possible and use statistical tests to show that chrF is only useful for edge cases and otherwise not really -> use plots where possible and otherwise add abc annotations via ggplot if necessary
% explain that papers like volatility one might be making claims based on weaker models that could be fixed by using larger models with better training -> hard to say exactly but based on our paraphrases which are non-adversarial and non-targetted, this seems to be the case but also no hard conclusion -> depends on also which model they used
% we can say chrF is not a measure of semantic equality but might still be correlated to semantic similarity
% list hypotheses and how some were refuted by results
% make less confident conclusion on relationship between back-translation and translation consistency -> could also be linked to other differences between models
% there are many possible contributions to better performance compared to non-SOTA -> could be langids to help prevent language mixing (observed in non-SOTA), could be backtranslation for syntactical and lexical diversity which contributes to paraphrase robustness
% provide some minor criticisms where appropriate -> such as some sentences being over-paraphrased

% 7. Conclusions
% main conclusions from this on isometry -> could be linked either to test data better fitting to WMT19 or otherwise massive backtranslation which provides some regularizing effect for FAIR SOTA model
% summarize everything and list contributions of this paper

% 8. Recommendations
% WMT16 transformer optimized for short sentences -> might make comparison also not good -> WMT16 vs. WMT19 issue which is a limitation of this research -> better to have used a WMT19 variant which was less performant
% paraphrase detection models are not commutative -> not logical in that sense -> maybe cosine similarity comparisons could help but might affect performance
% compare chrF correlation to BLEU to see if it is indeed better
% describe processes that worked and did not work -> talk about all the hurdles and show some bad examples when they occurred -> summarized below in logs
% limitation of system is that WMT19 might be more advantageous to FAIR model, which might have led to bad results in general for non-SOTA model
% Criticism of using WMT19 model is that it may be biased towards the WMT19 test set compared to the model trained on WMT16

% Post-paper formatting:
% make usage of colons consistent
% do thorough language check on overleaf or otherwise to ensure proper grammar and spelling
% consider adding page numbers for easier reading

\clearpage
% references
\bibliography{bibtex}
\bibliographystyle{acl_natbib}

% end document
\end{document}